% Copyright (C) 2017 Daniel Page <csdsp@bristol.ac.uk>
%
% Use of this source code is restricted per the CC BY-SA license, a copy of
% which can be found via http://creativecommons.org (and should be included 
% as LICENSE.txt within the associated archive or repository).
%
% You can find some documentation in the associated repo. hosted at
%
% https://github.com/danpage/uob-dissertation

\documentclass[ % the name of the author
                    author={Daniel Page},
                % the name of the supervisor
                supervisor={Dr. Andrew Calway},
                % the degree programme
                    degree={MEng},
                % the dissertation    title (which cannot be blank)
                     title={Some Structural Guidelines for CS Posters},
                % the dissertation subtitle (which can    be blank)
                  subtitle={},
                % the dissertation     type
                      type={enterprise},
                % the year of submission
                      year={2014},
                % the poster board number
                     board={} ]{poster}

\begin{document}

% =============================================================================

\begin{frame}{} 

\vfill

\begin{columns}[onlytextwidth]
  \begin{column}[t]{1.0\textwidth-0.0cm}
  \begin{block}{\Large 1. Introduction}
  It is hard to give generic advice about what form your poster should 
  take, since each project relates to a different topic and each student
  will be at a different stage wrt. completeness.  Therefore, the best 
  approach is to focus on the underlying aim of the poster presentation:
  essentially the intention is for you to get early, objective opinions
  about your work and then (ideally) improve it as a result.

  With this in mind, one idea is to

  \begin{enumerate}
  \item think about how to explain your project to someone, and questions
        you might want an answer to or opinion on,
  \item consider the poster as a set of slides, which support an elevator
        pitch~\cite{poster:pitch}
        for either the technical and/or business plan part,
        then
  \item focus the poster content on the part you feel you need the most
        input on.
  \end{enumerate}

  \noindent
  Another approach is to adopt standard advice about developing research 
  posters~\cite{poster:style}, 
  then produce a stand-alone result that summarises your project (see
  examples on walls throughout the MVB).  Either way, the blocks below 
  attempt to outline some potential examples of content.
  \end{block}
  \end{column}
\end{columns}

\vfill

\begin{columns}[onlytextwidth]
  \begin{column}[t]{0.5\textwidth-1.0cm}
  \begin{block}{\Large 2. Project Outline}
  Example content could follow initial specification, and might include:

  \begin{itemize}
  \item an outline of the problem context,
  \item a description of the central challenge, 
  \item an overview of the direction (within the possible options) you 
        have opted to take,
        and
  \item a concrete list of aims and objectives.
  \end{itemize}
  \end{block}
  \end{column}

  \begin{column}[t]{0.5\textwidth-1.0cm}
  \begin{block}{\Large 3. Business Plan/Research Proposal}
  Example content might include:

  \begin{itemize}
  \item identification and analysis of a market,
  \item proposed product/service portfolio,
  \item ideas about development and protection of IP,
  \item proposed company organisation,
        and
  \item estimates for start-up and recurrent costs.
  \end{itemize}
  \end{block}
  \end{column}
\end{columns}

\vfill

\begin{columns}[onlytextwidth]
  \begin{column}[t]{0.5\textwidth-1.0cm}
  \begin{block}{\Large 4. Preliminary Results~\cite{poster:xkcd}}
  \vspace{4ex}
  \centerline{\includegraphics[scale={15.0}]{image/example.pdf}}
  \vspace{4ex}
  \end{block}
  \end{column}
  \begin{column}[t]{0.5\textwidth-1.0cm}
  \begin{block}{\Large 5. Progress and Status}
  Example content might include:

  \begin{itemize}
  \item a list of complete and incomplete aims and objectives,
  \item a list of open questions or problems,
        and
  \item your plan for completing the project, inc. required deliverables.
  \end{itemize}
  \end{block}
  \end{column}
\end{columns}

\vfill

\begin{columns}[onlytextwidth]
  \begin{column}[t]{1.0\textwidth-0.0cm}
  \begin{block}{\Large 6. References}
  \printbibliography
  \end{block}
  \end{column}
\end{columns}

\vfill

\end{frame}

% =============================================================================

\end{document}



